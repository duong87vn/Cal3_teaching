\documentclass[12pt]{amsart}
\usepackage{amsmath}
\usepackage[margin=1in]{geometry}
\linespread{1.6}
\usepackage{dirtytalk}
\newtheorem{theorem}{Theorem}
\newtheorem{defn}{Definition}

\author{DUONG THAN} 
\title{Calculus 3 }
\begin{document}
\maketitle
\large

\textbf{Text Book} Calculus: Concept $\&$ Context. Author: James Sterwart. \\

\section{Chapter 8: Infinite Sequences and Series}

\subsection{8.1: Sequences}

\begin{enumerate}

	\item Definition: a sequence can be thought of as a list of numbers written in a definite order: 
	\[
	a_1,a_2,\dots,a_n,\dots
	\]
	where, $a_1$ is the first tern; $a_2$ is the second term; and in general $a_n$ is the $n^{th}$ term. \\
	Notation: The sequence $\{ a_1, a_2,\dots\}$ is also denoted by 
	\[
	\{a_n\} \quad \text{or} \quad \{a_n\}_{n=1}^{\infty}
	\]
 	\item Example
		\begin{enumerate}
			\item $\Big\{\dfrac{n}{n+1} \Big\}_{n=1}^{\infty}$
			
			\vspace{2in}
			
			\item $\Big\{\dfrac{(-1)^n(n+1)}{3^n} \Big\}_{n=1}^{\infty}$
			
			\vspace{2in}
			
			\item $\Big\{cos\dfrac{n\pi}{6}\Big\}_{n=1}^{\infty}$
		
		\end{enumerate}
		
		\vspace{4in}
		
	\item The limit of sequence: A sequence $\{ a_n\}$ has the limit $L$ and we write 
	\[
	\lim_{n \to \infty} a_n = L \quad or \quad a_n \to L \quad as \quad n \to \infty
	\] 
	 if we can make the terms $a_n$ as close to L as we like by taking n sufficiently large. If $lim_{n \to \infty}$ exists, we say the sequence \textbf{converges} (or is \textbf{convergent
	 }). Otherwise, we say the sequence \textbf{diverges} or ( is \textbf{divergent} ).
	
	\item Theorem: \\
	If $\lim_{x \to \infty} f(x) = L$ and $f(n) = a_n$ when n is an integer, then $\lim_{n \to \infty} a_n  = L.$
		Example: we know that $\lim_{x \to \infty} \dfrac{1}{x^r} = 0$ when $r > 0$, we have 
		\[
		\lim_{n \to \infty} \dfrac{1}{n^r} = 0 \quad when \quad r > 0
		\]
	
	\item Squeeze Theorem for Sequences \\
	If $a_n \leq b_n \leq c_n$ for $n \geq n_0$ and $\lim_{n \to \infty}a_n = \lim_{n \to \infty}c_n = L$, then $\lim_{n \to \infty}b_n = L$.
	
	\item Limit Laws for Sequences:\\
		If $a_n$ and $b_n$ are convergent sequences and  $c$ is a constant, then: \\
			\begin{enumerate}
				\item $\lim_{n \to \infty}(a_n + b_n) = \lim_{n \to \infty}a_n + \lim_{n \to \infty}b_n$.
				\item $\lim_{n \to \infty}a_n - b_n) = \lim_{n \to \infty}a_n - \lim_{n \to \infty}b_n$.	
				\item $\lim_{n \to \infty}ca_n = c\lim_{n \to \infty}a_n$. \quad \quad $\lim_{n \to \infty}c = c.$
				\item $\lim_{n \to \infty}(a_nb_n) = \lim_{n \to \infty}a_n \times \lim_{n \to \infty}b_n$.
				\item $ \lim_{n \to \infty} \dfrac{a_n}{b_n} = \dfrac{\lim_{n \to \infty}a_n}{\lim_{n \to \infty}b_n}.$ if $\lim_{n \to \infty}b_n \neq 0.$
				\item $\lim_{n \to \infty}a_n^p = \Bigg[ \lim_{n \to \infty}a_n\Bigg]^p$	if $p > 0$ and $a_n > 0$.
				\end{enumerate}
		
	
	\item Exercises: 
		\begin{enumerate}
		
			\item Use the squeeze theorem to prove: \\
			if $\lim_{n \to \infty}|a_n| = 0$, then $\lim_{n \to \infty}a_n = 0.$ \\
			
			\item Find $\lim_{n \to \infty}\dfrac{n}{n=1}$
			
			
			\item Calculate $\lim_{n \to \infty}\dfrac{\text{ln} n}{n}.$
			
			\item Evaluate $\lim_{n \to \infty}\dfrac{(-1)^n}{n}$ if it exists.
			
			
			
			
		\end{enumerate}
	\item Theorem: \\
			If $\lim_{n \to \infty}a_n = L$, and the function $f$ is continuous at L, then 
			\[
			\lim_{n \to \infty} f(a_n) = f(L)
			\]
		Example: Find $\lim_{n \to \infty} \text{sin}\Bigg(\dfrac{\pi}{n}\Bigg)$.\\
		
	\item Theorem: \\
		The sequence $r^n$ is convergent if $-1 < r \leq 1$ and divergent for all other values of $r$.
		\[
		\lim_{n \to \infty} r^n = \begin{cases} 0 \quad if \quad -1 < r < 1 \\
									1 \quad if \quad r = 1 \\
						\end{cases}
		\]
	
	\item Theorem 
	
	\item Theorem 
	
	\item Increasing and Decreasing sequences: \\
		A sequence $a_n$ is called \textbf{increasing} if $a_n < a_{n+1}$ for all $n \geq 1$, that is, $a_1 < a_2 < a_3 \dots.$ It is called \textbf{decreasing} if $a_n > a_{n+1}$ for all $n \geq 1$. A sequence is \textbf{monotonic} if it is either increasing or decreasing. \\
		
		Example: \\
			\begin{enumerate}
				\item The sequence $\Bigg\{ \dfrac{3}{n+5}\Bigg\}$ ... 
				
				\item Show that the sequence $a_n = \dfrac{n}{n^2 + 1}$ is decreasing.
				
			\end{enumerate} 
	
	\item Bounded sequence:\\
		A sequence $a_n$ is bounded above if there is a number $M$ such that 
		\[
		a_n \leq M \quad \text{for all}  \quad n \geq 1
		\]
	It is bounded below if there is a number $m$ such that 
	\[
	m \leq a_n  \quad \text{for all}  \quad n \geq 1
	\]
	
	If it is bounded above and below, then $a_n $ is a bounded sequence.
	
	\item Monotonic Sequence Theorem: Every bounded monotonic sequence is convergent. \\
		Example: \\
	
	\item Practices: problem 4, 5, 8, 11, 15, 21, 25, 49, 51, 37, 38, 50, 52\\
	
	\item Homework 12, 18, 20, 30, 22, 42, 50, 52 page 562. 
\end{enumerate}
\newpage
\subsection{8.2: Series}
\begin{enumerate}
	\item Series \\
		If we try to add an infinite sequence $\{a_n\}_{n=1}^{\infty}$ we get \\
		\[
		a_1 + a_2 + a_3 + \dots + a_n + \dots
		\]
which is called an \textbf{infinite series} or just \textbf{a series} and is denoted
	\[
	\sum_{n=1}^{\infty}a_n \quad \quad or \quad \quad \sum a_n
	\]
	
	Example: \\
	
	\textbf{Partial Sums}
	
	\begin{equation} \label{eq1}
		\begin{split}
 s_1 & = a_1 \\
 s_2 & = a_1 + a_2 \\
 s_3 & = a_1 + a_2 + a_3 \\
 		\end{split}
	\end{equation} \\
In general, 
\[
s_n = a_1 + a_2 + a_3 + \dots + a_n = \sum_{i=1}^{n}a_i
\]
These partial sums form a new sequence $\{ s_n\}$, which may or may not have a limit. 

\item Definition: Given a series $\sum_{n=1}^{\infty} a_n = a_1 + a_2 + \dots ,$ let $s_n$ denote its nth partial sum:
\[
s_n = \sum_{i=1}^{n}a_i = a_1 + a_2 + \dots + a_n
\]

If the sequence $\{ s_n\}$ is convergent and $\lim_{}^{}s_n = s$ exists as a real number, then the series $\sum a_n$ is called convergent and we write: \\
\[
a_1 + a_2 + \dots + a_n + \dots = s \quad or \quad \sum_{n=1}^{\infty} a_n = s
\]
The number s is called the \textbf{sum} of the series. If the sequence $\{ s_n\}$, is divergent, then the series is called divergent. 

Example: Geometric Series \\

Example: Harmonic Series \\

\item Theorem: \\
If the series $\{ a_n\}$ is convergent, then $\lim_{n \to \infty}a_n = 0$.\\

Proof: \\ 

Note1: the limit of $\{ s_n\}$ is s, and the limit of $\{ a_n\}$ is 0. 

Note 2: The converse of this theorem is not true. For example, the harmonic series is divergent while the limit of the sequence is 0.\\


\item The test for Divergence: \\
	If $\lim_{n \to \infty}a_n $ does not exist or if $\lim_{n \to \infty}a_n \neq 0$, then the series $\sum_{}^{} a_n $ is divergent. 
	
	Example: \\
	
\item Theorem: \\
	If $\sum a_n$ and $\sum b_n$ are convergent series, then so are the series $\sum ca_n$ (where c is a constant), $\sum (a_n + b_n)$, and $\sum (a_n - b_n)$, and 
		\begin{enumerate}
		
			\item $\sum_{n=1}^{\infty} ca_n = c \sum_{n=1}^{\infty}a_n$
			
			\item $ \sum_{n=1}^{\infty}(a_n+b_n) = \sum_{n=1}^{\infty}a_n + \sum_{n=1}^{\infty}b_n $
			
			\item $\sum_{n=1}^{\infty}(a_n - b_n) = \sum_{n=1}^{\infty}a_n - \sum_{n=1}^{\infty}b_n$
		\end{enumerate}	
		
\item Practices: 13, 30, 17, 31, 37, 41 page 572 \\ 

\item Homework: 14, 28, 16, 36, 32, 34, 42 page 573		
\end{enumerate}

\newpage
\subsection{8.3: The Integral and Comparison Tests; Estimating Sums}

\newpage
\subsection{8.4: Other Convergence Tests}

\newpage
\subsection{8.5: Power Series}

\newpage
\subsection{8.6: Representations of Functions  as Power Series}

\newpage
\subsection{8.7: Taylor and Maclaurin Series}

\newpage
\section{Chapter 9: Vector and the Geometry of Space}

\subsection{9.1: Three-Dimensional Coordinate Systems}
	\begin{enumerate}
		\item Three coordinates axes (x,y,z) form three coordinate planes.\\
			xy plane, yz plane, xz plane. \\
			
			8 octants. The first octant is determined by the positive axes. \\
			
			Projection of a point onto planes. \\
			
			Example: Graphing equations \\
			 
			Example: Describe regions represented by equations\\
		\item Distance Formula in Three Dimensions \\
				The distance $|P_1P_2|$ between the points $P_1(x_1,y_1,z_1)$ and $P_2(x_2,y_2,z_2)$ is\\
				\[
				|P_1P_2| = \sqrt{(x_2  - x_1)^2 + (y_2 - y_1)^2 + (z_2 - z_1)^2}
				\]
				Example: \\
		\item Equation of a Sphere \\
				An equation of a sphere with center $C(h,k,l)$ and radius $r$ is 
				\[
				(x-h)^2 + (y-k)^2 + (z-l)^2 = r^2
				\]
				In particular, if the center is the origin O, then an equation of the sphere is 
				\[
				x^2 + y^2 + z^2 = r^2
				\]	
				
				Example: \\
				
				Example: \\
			\item Practices: 3, 11, 19, 31 \\
			
			\item Homework:  2, 12, 14, 26\\
	\end{enumerate}
\newpage
\subsection{9.2: Vectors}

\newpage
\subsection{9.3: The Dot Product}

\newpage
\subsection{9.4: The Cross Product}

\newpage
\subsection{9.5: Equations of Lines and Planes}

\newpage
\subsection{9.6: Functions and Surfaces}

\newpage
\subsection{9.7: Cylindrical and Spherical Coordinates}

\newpage
\section{Chapter 10: Vector Functions}


\subsection{10.1: Vector Functions and Space Curves}

\newpage
\subsection{10.2: Derivatives and Integrals of Vector Functions}

\newpage
\subsection{10.3: Arc Length and Curvature}

\newpage
\subsection{10.4: Motion in Space: Velocity and Acceleration}

\end{document}